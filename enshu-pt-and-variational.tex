\documentclass[11pt,pra,aps]{revtex4}
\usepackage{graphicx}
%\usepackage{overcite}
\usepackage{rotating}
\usepackage{array}
\usepackage{amsmath}
\usepackage{multirow}
\usepackage{setspace}
\usepackage{braket}
\usepackage{epstopdf}
\usepackage{moreverb}

\usepackage{color}                            
                                              
\newcommand{\red}[1]{\textcolor{red}{#1}}     
\newcommand{\blue}[1]{\textcolor{blue}{#1}}

%%\renewcommand{\baselinestretch}{2.0}

\renewcommand{\thefigure}{S\arabic{figure}}
\renewcommand{\thetable}{S\arabic{table}}

\begin{document}
\title{摂動論と変分法}
\author{齋藤 雅明 \\ 量子化学研究室 \\ email: masa.saitow@chem.nagoya-u.ac.jp}

\maketitle

\noindent
{\bf 問題A.} 水素原子のSchr\"odinger方程式は、$r$のみに依存する動径方程式と、$\theta$及び$\phi$に依存する角度方程式とに分離され、エネルギーは動径方程式を解くことにより決定される。動径方程式におけるHamiltonianは原子単位で
\begin{align}
  H_\text{radial}=-\frac{\hbar^2}{2}\frac{\partial}{\partial r}\left(r^2 \frac{\partial}{\partial r}\right) - \frac{1}{r} \label{eq1}
\end{align}
と与えられる。ここで以下の問いに答えよ。

\noindent
{\bf 問い1} 式(\ref{eq1})の固有値を変分法で近似的に求める。試行関数として$\Theta(\alpha)=\exp(-\alpha r^2)$を用いた場合、エネルギー期待値が
\begin{align}
  E[\alpha] &= \frac{\langle\Theta(\alpha)|H_\text{radial}|\Theta(\alpha)\rangle}{\langle\Theta(\alpha)|\Theta(\alpha)\rangle} \\ \nonumber
            &= \frac{3}{2}\alpha - 2\sqrt{\frac{2\alpha}{\pi}}
\end{align}
となることを示せ。


\end{document}
