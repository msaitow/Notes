\documentclass{jlreq}

\usepackage{luatexja-fontspec}
\usepackage{amsmath,amssymb,amsfonts}

\begin{document}

\noindent
\textbf{問1.}\\


\noindent
\textbf{(a).}

次の特性多項式を解く。  
\begin{eqnarray}
    \det
    \begin{vmatrix}
        E_1-\epsilon & \lambda\Delta \\
        \lambda\Delta & E_2-\epsilon 
    \end{vmatrix}=0\Rightarrow \epsilon^2+\epsilon(-E_1-E_2)+(E_1E_2-\lambda^2\Delta^2)=0   
\end{eqnarray}  
2次方程式の解の公式から
\begin{eqnarray}
    \epsilon_{1,2}=\frac{E_1+E_2}{2}\pm\sqrt{\left(\frac{E_1-E_2}{2}\right)^2+\lambda^2\Delta^2}
\end{eqnarray}  
と求まる。

\noindent
\textbf{(b).}

\begin{eqnarray}
    \sqrt{a+bx}=\sqrt{a}+\frac{b}{2\sqrt{a}}x+\cdots
\end{eqnarray}  

であるから

\begin{eqnarray}
    \sqrt{\left(\frac{E_1-E_2}{2}\right)^2+\lambda^2\Delta^2}\approx\frac{E_1-E_2}{2}+\frac{\Delta^2\lambda^2}{E_1-E_2}
\end{eqnarray}

となり、故にエネルギーは

\begin{eqnarray}
    \epsilon_1\approx E_1+\frac{\Delta^2\lambda^2}{E_1-E_2}
\end{eqnarray}

\begin{eqnarray}
    \epsilon_2\approx E_2-\frac{\Delta^2\lambda^2}{E_1-E_2}
\end{eqnarray}

となる。

\noindent
\textbf{(c).}

エネルギー差は

\begin{eqnarray}
\delta E=\frac{2\Delta^2\lambda^2}{E_1-E_2}
\end{eqnarray}

となる。これが最大化されるのは$E_1$と$E_2$とが等しいときである。

\clearpage

\noindent
\textbf{問2.}\\

\noindent
\textbf{(a).} 

共鳴安定化エネルギーを計算するために、まずエチレンの全$\pi$電子エネルギーを求める。ハミルトニアンは
\begin{eqnarray}
    \mathbf{H}=
    \begin{pmatrix}
        \alpha & \beta \\
        \beta & \alpha
    \end{pmatrix}   
\end{eqnarray}  
である。Schr\"odinger方程式は
\begin{eqnarray}
    \mathbf{HC}=\epsilon\mathbf{C}
\end{eqnarray}  
であり、これを変形すると
\begin{eqnarray}
    %\mathbf{xC}=\frac{E-\alpha}{\beta}\mathbf{C}
    \mathbf{xC}=\lambda\mathbf{C}
\end{eqnarray}
となる。ここで
\begin{eqnarray}
    \mathbf{x}=
    \begin{pmatrix}
        0 & 1 \\
        1 & 0
    \end{pmatrix}
\end{eqnarray}
\begin{eqnarray}
\lambda=\frac{\epsilon-\alpha}{\beta}
\end{eqnarray}  
である。次の特性多項式を解く。
\begin{eqnarray}
    \det|\mathbf{x}-\lambda\mathbf{E}|=0
\end{eqnarray}  
以下の二つの固有値及び固有ベクトルを得る。

\noindent
\textbf{$\lambda=-1$の場合:}\\
\begin{eqnarray}
    \epsilon_1=\alpha+\beta
\end{eqnarray}  
\begin{eqnarray}
    \mathbf{C}_1=\frac{1}{\sqrt{2}}
    \begin{pmatrix}
        1 \\
        1
    \end{pmatrix}   
\end{eqnarray}

\noindent   
\textbf{$\lambda=1$の場合:}\\ 
\begin{eqnarray}
    \epsilon_2=\alpha-\beta
\end{eqnarray}  
\begin{eqnarray}
    \mathbf{C}_2=\frac{1}{\sqrt{2}}
    \begin{pmatrix}
         1 \\
        -1
    \end{pmatrix}   
\end{eqnarray}
故にエチレンの全$\pi$電子エネルギーは
\begin{eqnarray}
    E_\text{ethylene}=2\alpha+2\beta
\end{eqnarray}  
となる。共鳴安定化エネルギーは
\begin{eqnarray}
    \Delta E_\text{RSE}=N\alpha+\frac{4\beta}{\sin(\pi/N)}-\frac{N}{2}\left(2\alpha+2\beta\right)=\beta\left(\frac{4}{\sin(\pi/N)}-N\right)
\end{eqnarray}
となる。

\noindent
\textbf{(b).}

$N$が大きい極限では$\pi/N\rightarrow 0$である。そこで
\begin{eqnarray}
    \sin(x)=x-\frac{x^2}{2!}+\frac{x^3}{3!}-\cdots
\end{eqnarray}
であるから
\begin{eqnarray}
    \sin\left(\frac{\pi}{N}\right)\approx\frac{\pi}{N}
\end{eqnarray}
故に
\begin{eqnarray}
    \Delta E_\text{RSE}\approx\left[\frac{4}{\pi}-1\right]\beta N=0.2732N\beta
\end{eqnarray}
となる。$\Delta E_\text{RSE}$は正確に$N$に正比例するから、厳密に示量性といえる。

\clearpage

\noindent
\textbf{問3.}\\

ハミルトニアンは
\begin{align}
    H=-\frac{\hbar^2}{2m}\frac{d^2}{sx^2}+\frac{1}{2}m\omega^2x^2
\end{align}
と与えられ、$\beta$をパラメータとしたエネルギー積分を考える。
\begin{align}
    \langle H\rangle_\beta\equiv\frac{\langle\tilde{\phi}|H|\tilde{\phi}\rangle}{\langle\tilde{\phi}|\tilde{\phi}\rangle}
\end{align}
分母は
\begin{align}
    \langle\tilde{\phi}|\tilde{\phi}\rangle=\int^\infty_{-\infty}e^{-2\beta|x|} dx=2\int^\infty_{0}e^{-2\beta x} dx = -\frac{1}{\beta}[e^{-2\beta x}]^\infty_0=\frac{1}{\beta}
\end{align}
となる。分子の微分を逐次計算していく。
\begin{align}
    \frac{d}{dx}e^{-\beta |x|}=-\beta (2\theta(x)-1)e^{-2\beta |x|}
\end{align}
\begin{align}
    \frac{d^2}{dx^2}e^{-\beta |x|}=-\beta (2\delta(x)-1)e^{-2\beta |x|}+\beta^2 (2\delta(x)-1)^2 e^{-2\beta |x|}=-\beta (2\delta(x)-1)e^{-2\beta |x|}+\beta^2 e^{-2\beta |x|}
\end{align} 
ここで$(2\theta(x)-1)^2=1$となることを使った。またポテンシャル部分の積分は
\begin{align}
    \int^\infty_{-\infty} x^2 e^{-2\beta|x|}dx=2\int^{\infty}_0 x^2 e^{-2\beta x}dx=\frac{1}{2\beta^3}
\end{align}
となる。また
\begin{align}
    \int^\infty_{-\infty} \delta(x) e^{-2\beta|x|}dx=1
\end{align}
だから、以上より
\begin{align}
    \langle\tilde{\phi}|\frac{d^2}{dx^2}|\tilde{\phi}\rangle=-\beta
\end{align}
となり
\begin{align}
    \langle\tilde{\phi}|H|\tilde{\phi}\rangle=-\frac{\hbar^2}{2m}(-\beta)+\frac{1}{2}m\omega^2\frac{1}{2\beta^3}
\end{align}
故に
\begin{align}
    \frac{\langle\tilde{\phi}|H|\tilde{\phi}\rangle}{\langle\tilde{\phi}|\tilde{\phi}\rangle}=\frac{\hbar^2}{2m}\beta^2+\frac{m\omega^2}{4\beta^2}
\end{align}
を得る。これを$\beta$について微分すると
\begin{align}
    &\frac{\partial}{\partial\beta}\frac{\langle\tilde{\phi}|H|\tilde{\phi}\rangle}{\langle\tilde{\phi}|\tilde{\phi}\rangle}=\frac{\hbar^2}{m}\beta-\frac{m\omega^2}{2\beta^3}\rightarrow 0 \\
    &\Rightarrow \beta^2=\frac{1}{\sqrt{2}}m\omega
\end{align}
となる。得られた$\beta$をエネルギー積分に代入すると
\begin{align}
    \langle H\rangle^\text{min}=\frac{\hbar^2}{2m}\frac{1}{\sqrt{2}}m\omega+\frac{m\omega^2}{4} \frac{\sqrt{2}\hbar}{m\omega}=\frac{1}{\sqrt{2}}\hbar\omega=0.707\hbar\omega
\end{align}
となる。正確な値である$0.5\hbar\omega$の約41\%高い値であることが分かる。
\end{document}
	
