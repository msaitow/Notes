\documentclass[11pt,pra,aps]{revtex4}
\usepackage[dvipdfmx]{graphicx}
\usepackage{float}
\usepackage{rotating}
\usepackage{array}
\usepackage{amsmath}
\usepackage{multirow}
\usepackage{setspace}
\usepackage{braket}
\usepackage{epstopdf}
\usepackage{moreverb}

\usepackage{color}                            
                                              
\newcommand{\red}[1]{\textcolor{red}{#1}}     
\newcommand{\blue}[1]{\textcolor{blue}{#1}}

\newcommand{\boxz}[1]{\boxed{\phantom{\text{#1}}}}
\newcommand{\boxa}[1]{\boxed{\phantom{}}}

%%\renewcommand{\baselinestretch}{2.0}

\renewcommand{\thefigure}{S\arabic{figure}}
\renewcommand{\thetable}{S\arabic{table}}

\begin{document}
\title{摂動論と変分法}
\author{齋藤 雅明 \\ 量子化学研究室 \\ email: masa.saitow@chem.nagoya-u.ac.jp}

\maketitle

\noindent
{\bf 問題1.} 以下の空欄を埋めよ。

% Levine 8.1
\noindent
{\bf 1.1} Li原子の基底状態エネルギーに対して3つの変分計算が行われ、その結果-203.2 eV、-192.0 eV、-201.2 eVという結果が得られた。このことからLi原子の真の基底状態エネルギーは\boxz{-203.2 eV}よりも\boxz{低い}ことが分かる。

\noindent
{\bf 1.2} 水素原子のシュレーディンガー方程式は極座標系では、同系部分と角度部分とに分離される。エネルギーは角度方程式のみから決定され、角度部分のハミルトニアンは以下の式で与えられる。
\begin{align}
  H = -\frac{\hbar^2}{2m_e r^2}\frac{d}{dr}\left(r^2\frac{d}{dr}\right)-\frac{e^2}{4\pi\epsilon_0 r}
\end{align}
この最低エネルギー固有値$E_\text{exact}$は原子単位で\boxz{-0.5}と与えられる。試行関数としてガウス型関数
\begin{align}
  \psi_\text{trial}(\alpha;r)=N\exp\left(-\alpha r^2\right) \ (\alpha\text{は変分パラメータ})
\end{align}
を用いた場合はエネルギー$E_\text{trial}$は原子単位で\boxz{}と計算される。これは変分原理を満足しており、$E_\text{exact}$\boxz{$\leq$}$E_\text{trial}$という大小関係が成立する。

\noindent
{\bf 問題2.} 以下の問いに答えよ。

% Levine 8.4
\noindent
{\bf 2.1} 次のような長さ$l$の無限に高い一次元井戸型ポテンシャル
\begin{align}
  V(x)=\left\{
  \begin{array}{cc}
    0 & (0 \leq x \leq l) \\
    \infty & (\text{それ以外})
  \end{array}
  \right.
\end{align}
中の粒子のエネルギー計算に以下の規格化された試行関数
\begin{align}
  \psi_\text{trial}(x) = \sqrt{\frac{3}{l^3}}x
\end{align}
を用いるとき、エネルギー期待値は0となり、真のエネルギーよりも低く見積もられる。この理由を述べよ。

% McQuarrie 7.8
\noindent
{\bf 2.2} 3次元球対称調和振動子の基底状態エネルギーを考える。$\alpha$を変分パラメータとする以下の二つの試行関数を用いた場合、どちらの方がより真の解に近い結果を与えるか。理由も併せて延べよ。
\begin{align}
  \psi_\text{trial}(r)&=N\exp\left(-\alpha r\right) \\
  \psi_\text{trial}(r)&=N\exp\left(-\alpha r^2\right)
\end{align}
ここでハミルトニアン演算子は
\begin{align}
  H=-\frac{\hbar^2}{2\mu r^2}\frac{d}{dr}\left(r^2\frac{d}{dr}\right)+\frac{k}{2}r^2
\end{align}
である。

% Levine 9.8
\noindent
{\bf 問題3.} プロトンの正電荷が半径$10^{-3}$ cmの球内に一様に分布していると仮定する。このとき、1次の摂動論を用いて、水素原子基底状態エネルギーにおける変化を計算せよ。有限サイズの核内に存在する電子が感じる摂動ポテンシャルは
\begin{align}
  V(r) = -\frac{eQ(r)}{4\pi\epsilon_0 r}
\end{align}
と与えられる。ここで$Q(r)$は半径$r$の球内に存在するプロトン電荷である。積分計算には公式集を用いても良い。また実際の積分範囲を考慮した上での積分計算の単純化を行っても良い。ただし単純化を行った場合には、なぜその単純化が有効かを述べること。

% Levine 9.21
\noindent
{\bf 問題4.} 以下の問いに答えよ。

\noindent
{\bf 4.1} 一辺の長さ$l$で、点$x=0$, $y=0$に原点を持つ無限に高い二次元箱型ポテンシャル中の粒子について考える。波動関数及びエネルギーを示せ。
    
\noindent
{\bf 4.2} 4.1の粒子が以下の摂動にさらされた際の1次のエネルギー変化を示せ。
\begin{align}
  V(x,y) = \left\{
  \begin{array}{ll}
    b & (\frac{1}{4}l\leq x\leq\frac{3}{4}l) \ \text{and}\ (\frac{1}{4}l\leq y\leq\frac{3}{4}l) \\
    0 & (\text{それ以外})
  \end{array}
  \right.
\end{align}
ここで$b$は定数である。

% JJ Sakurai 5.3
\noindent
{\bf 4.3} 4.1の粒子が以下の摂動にさらされた際の基底状態に対する1次のエネルギー変化を示せ。
\begin{align}
  V(x,y) = \left\{
  \begin{array}{ll}
    cxy & (0\leq x\leq l) \ \text{and}\ (0\leq y\leq l) \\
    0 & (\text{それ以外})
  \end{array}
  \right.
\end{align}
ここで$c$は定数である。また第一励起状態に対する1次のエネルギー変化を縮退がある場合の摂動論を用いて示せ。

% Levine 9.30
\noindent
{\bf 問題5.} 以下の言明の正誤を答えよ。
    
\noindent
{\bf 5.1} 時間依存シュレーディンガー方程式の解の任意の線形結合もまた、その時間依存シュレーディンガー方程式の解となる。

\noindent
{\bf 5.2} 時間非依存シュレーディンガー方程式の解の任意の線形結合もまた、その時間非依存シュレーディンガー方程式の解となる。

\noindent
{\bf 5.3} 縮退が存在しない場合の1時摂動エネルギーの式$E^{(1)}=\langle\psi^{(0)}|V|\psi^{(0)}\rangle$は基底状態にのみ適用可能である。ここで$V$は摂動ポテンシャル、$\psi^{(0)}$は無摂動波動関数である。

\noindent
{\bf 5.4} He原子の厳密な基底状態波動関数は$f(1)f(2)$という形式で記述される。ここで$f(i)$は$i$番目の電子座標に関する関数である。
    

\end{document}
