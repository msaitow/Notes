\documentclass{jlreq}

\usepackage{luatexja-fontspec}
\usepackage{amsmath,amssymb,amsfonts}

\begin{document}

\noindent
\textbf{問1.}\\

\noindent
化学結合の生成は、次の二準位系ハミルトニアン行列で大雑把にモデル化できる。

\begin{eqnarray}
\mathbf{H}(\lambda)=
\begin{pmatrix}
E_1 & \lambda\Delta \\ 
\lambda\Delta & E_2
\end{pmatrix}
\end{eqnarray}

ここで$E_1$および$E_2$は、結合生成に関与=する原子軌道のエネルギー準位であり、$\lambda$は原子軌道間に働く相互作用(摂動)の強さである。このとき、以下の問い答えよ。\\

\noindent
(a). ハミルトニアン行列$\mathbf{H}(\lambda)$のエネルギー固有値$\epsilon_1$及び$\epsilon_2$を相互作用$\lambda$の関数として計算せよ。\\
\noindent
(b). 問い(a)で得られたエネルギーは、$\lambda \ll |E_1-E_2|$と仮定出来る場合にはどうなるか。$\lambda$に関して2次の項まで考えよ。\\
\noindent
(c). 問い(b)で得られたネルギーは、それぞれ結合性、反結合性軌道のエネルギーである。このとき、これらのエネルギーギャップが最も大きくなるのはどのような場合か。\\

\noindent
\textbf{ヒント:} $\sqrt{a+bx}$を$x$に関してマクローリン展開し、$x$の1次の項まで考える。これに$\lambda^2$などを適宜代入する。

\end{document}
	