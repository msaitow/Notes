\documentclass[11pt,pra,aps]{revtex4}
\usepackage{graphicx}
%\usepackage{overcite}
\usepackage{rotating}
\usepackage{array}
\usepackage{amsmath}
\usepackage{multirow}
\usepackage{setspace}
\usepackage{braket}
\usepackage{epstopdf}
\usepackage{moreverb}

\usepackage{color}                            
                                              
\newcommand{\red}[1]{\textcolor{red}{#1}}     
\newcommand{\blue}[1]{\textcolor{blue}{#1}}

%%\renewcommand{\baselinestretch}{2.0}

\renewcommand{\thefigure}{S\arabic{figure}}
\renewcommand{\thetable}{S\arabic{table}}

\begin{document}
\title{摂動論と変分法}
\author{齋藤 雅明 \\ 量子化学研究室 \\ email: masa.saitow@chem.nagoya-u.ac.jp}

\maketitle

\noindent
{\bf 問題A.} 水素原子のSchr\"odinger方程式は、$r$のみに依存する動径方程式と、$\theta$及び$\phi$に依存する角度方程式とに分離され、エネルギーは動径方程式を解くことにより決定される。動径方程式におけるHamiltonianは原子単位で
\begin{align}
  H_\text{radial}=-\frac{\hbar^2}{2}\frac{\partial}{\partial r}\left(r^2 \frac{\partial}{\partial r}\right) - \frac{1}{r} \label{eq1}
\end{align}
と与えられる。ポテンシャルが$r$のみに依存する球対称性を持つために、角度方程式は自由粒子の場合と同様となる。波動関数の角度部分は球面調和関数となる。ここで以下の問いに答えよ。

\noindent
{\bf 問い1} 式(\ref{eq1})の固有値を変分法で近似的に求める。試行関数として$\Theta_\alpha(r)=\exp(-\alpha r^2)$を用いた場合、エネルギー期待値が
\begin{align}
  E[\alpha] &= \frac{\langle\Theta_\alpha|H_\text{radial}|\Theta_\alpha\rangle}{\langle\Theta_\alpha|\Theta_\alpha\rangle} \\ \nonumber
            &= \frac{3}{2}\alpha - 2\sqrt{\frac{2\alpha}{\pi}} \label{eq2}
\end{align}
となることを示せ。また$\alpha$を変分パラメータとして、式(\ref{eq2})を最小化すると
\begin{align}
  E_\text{min}=-\frac{4}{3\pi}=-0.424\cdots
\end{align}
となることを示せ。これは真の値-0.5に対する上限となっている。

\noindent
{\bf 問い2} 問い1と同様の計算を試行関数$\Theta_\alpha(r)=\exp(-\alpha r)$を用いて行い、問い1の結果と比較せよ。

\noindent
{\bf 問い3} 水素原子に対してz軸方向の垂直電場
\begin{align}
  V(z)=Fz \label{eq:Vz}
\end{align}
が印加された場合を考える。球座様では式(\ref{eq:Vz})は
\begin{align}
  V(r,\theta)=Fr\cos\theta \label{eq:Vrtheta}
\end{align}
と変換される。式(\ref{eq:Vrtheta})が1$s$状態にある水素原子に印加された場合のエネルギー変化を一次の摂動論で求めよ。1$s$波動関数は
\begin{align}
  \Psi_{1s}=\frac{1}{\sqrt{\pi}} \exp(-r)
\end{align}
と与えられる。

\noindent
{\bf 問い4} 問い3と同様の計算を2$p_z$状態にある水素原子に対して計算せよ。2$p_z$波動関数は
\begin{align}
  \Psi_{2p_z}=\frac{1}{4\sqrt{2\pi}} r \exp(-\frac{r}{2})
\end{align}
と与えられる。

\noindent
{\bf 問題B.} 幅aの無限に高い箱型ポテンシャル内の自由粒子の量子状態について考える。ここで波動関数は$0\geq x\geq a$の範囲でのみ0でない値を持つこととする。

\noindent
{\bf 問い5} 問題Bで与えられている境界条件の下で一次元自由粒子のSchr\"odinger方程式を解き、エネルギーと波動関数を求めよ。

\noindent
{\bf 問い6} 問い5で得られた波動関数に対して、$V(x)=\frac{V_0}{a}x$と与えられる摂動が印加された場合の一次のエネルギー変化を求めよ。

\noindent
{\bf 問い7} 問題Bの箱型ポテンシャルを3次元に拡張する。半径aの球形の箱に閉じ込められた粒子の動径波動関数を考える。水素原子の場合と同様に、ポテンシャルは角度にのみ依存するために、エネルギーは動径方程式のみから決定され、波動関数の角度部分は球面調和関数となる。基底状態波動関数に対する動径方程式のHamiltonianは
\begin{align}
  H_\text{radial}=-\frac{\hbar^2}{2}\frac{\partial}{\partial r}\left(r^2 \frac{\partial}{\partial r}\right) \label{eq:freep}
\end{align}    
と与えられる。ここで動径波動関数を
\begin{align}
  \Psi_\text{radial}(r)=\frac{\Phi(r)}{r}
\end{align}
と置き、$\Phi$に関する方程式を求めよ。またこれが一次元箱型ポテンシャルのSchr\"odinger方程式と一致することを示せ。

\noindent
{\bf 問い8} 式(\ref{eq:freep})で与えられるHamiltonianの固有関数の一般解は
\begin{align}
  \Psi_\text{radial}(r)=A\frac{\cos\alpha r}{r} + B\frac{\sin\alpha r}{r} \label{eq:gensol}
\end{align}
と与えられる。式(\ref{eq:gensol})を用いて、$\Psi(a)=0$及び$\Psi(0)\neq\infty$という境界条件を満たす球形箱型ポテンシャル内の粒子の基底状態波動関数及び基底状態エネルギーを求めよ。

\noindent
{\bf 問い9} 問い8で得られた球形箱型ポテンシャルの基底状態エネルギーに対して、$\Theta(r)=(a-r)^2$という試行関数を用いた場合のエネルギーを求めよ。これは変分パラメータを含まないが、真の基底状態エネルギーの上限を与えるはずである。

\noindent
{\bf 問題C.} 3次元調和振動子の基底状態エネルギー及び基底状態波動関数を考える。球座標系では、この問題の動径Hamiltonianは
\begin{align}
  H_\text{radial}=-\frac{\hbar^2}{2\mu}\frac{\partial}{\partial r}\left(r^2 \frac{\partial}{\partial r}\right) + \frac{1}{2}kr^2\label{eq:3dp}
\end{align}    
と与えられる。

\noindent
{\bf 問い10} 式(\ref{eq:3dp})のHamiltonianで記述される系の基底状態エネルギーを、以下の試行関数を用いて計算せよ。
\begin{align}
  \Theta_\alpha(r)=\exp(-\alpha r^2) \label{eq:3dp-gauss}
\end{align}
式(\ref{eq:3dp-gauss})において$\alpha$は変分パラメータである。得られたエネルギーが真の値$\frac{3}{2}\hbar\omega$の上限値であることを示せ。

\noindent
{\bf 問い11} 式(\ref{eq:3dp})のHamiltonianで記述される系の基底状態エネルギーを、以下の試行関数を用いて計算せよ。
\begin{align}
  \Theta_\alpha(r)=\exp(-\alpha r) \label{eq:3dp-slater}
\end{align}
式(\ref{eq:3dp-slater})において$\alpha$は変分パラメータである。得られたエネルギーが真の値$\frac{3}{2}\hbar\omega$の上限値であることを示せ。

\section{補助問題}

\noindent
{\bf 問い12} 問い3及び問い4の結果を用いて、垂直電場下での水素原子の基底状態エネルギーを計算する。この系のHamiltonmianは
\begin{align}
  H=-\frac{1}{2}\nabla^2-\frac{1}{r}+Fr\cos\theta=H_\text{hydrogen}+Fr\cos\theta
\end{align}
である。試行関数として
\begin{align}
  \Theta_{\alpha_1\alpha_2}(r,\phi,\theta)=\alpha_1\Psi_{1s}(r,\phi,\theta)+\alpha_2\Psi_{2p_z}(r,\phi,\theta)
\end{align}
を用い、$\alpha_1$及び$\alpha_2$を最適化し、基底状態エネルギーを電場強度Fの関数として計算せよ。また以下の関係を用いて良い。
\begin{align}
  &H_\text{hydrogen}\Psi_{1s}(r,\phi,\theta)=-\frac{1}{2}\Psi_{1s}(r,\phi,\theta) \\
  &H_\text{hydrogen}\Psi_{2p_z}(r,\phi,\theta)=-\frac{1}{8}\Psi_{1s}(r,\phi,\theta)
\end{align}

\noindent
{\bf 問い13} 式(\ref{eq:3dp})で与えられる3次元調和振動子動径Hamiltonianの基底状態波動関数を計算せよ。

\section{積分公式}

\begin{align}
  &\int_0^\infty x^n \exp(-ax) dx = \frac{n!}{a^{n+1}} \\
  &\int_0^\infty x^{2n} \exp(-ax^2) dx = \frac{1\cdot3\cdot5\cdots(2n-1)}{2^{n+1}a^n}\sqrt{\frac{\pi}{a}} \\
  &\int_0^\infty x^{2n+1} \exp(-ax^2) dx = \frac{n!}{2 a^{n+1}}
\end{align}


\end{document}
