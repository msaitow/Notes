\documentclass[11pt,pra,aps]{revtex4}
\usepackage{graphicx}
%\usepackage{overcite}
\usepackage{rotating}
\usepackage{array}
\usepackage{amsmath}
\usepackage{multirow}
\usepackage{setspace}
\usepackage{braket}
\usepackage{epstopdf}
\usepackage{moreverb}

\usepackage{color}                            
                                              
\newcommand{\red}[1]{\textcolor{red}{#1}}     
\newcommand{\blue}[1]{\textcolor{blue}{#1}}

%%\renewcommand{\baselinestretch}{2.0}

\renewcommand{\thefigure}{S\arabic{figure}}
\renewcommand{\thetable}{S\arabic{table}}

\begin{document}
\title{補助問題}
\author{齋藤 雅明 \\ 量子化学研究室 \\ email: masa.saitow@chem.nagoya-u.ac.jp}

\maketitle

\noindent
{\bf 問題A.} 球対称ポテンシャル中にある粒子のSch\"odinger方程式の基底量子状態について考える。
\begin{align}
  H_\text{radial}=-\frac{\hbar^2}{2mr^2}\frac{\partial}{\partial r}\left(r^2 \frac{\partial}{\partial r}\right) + V(r) \label{eq1}
\end{align}

\noindent
{\bf 問い1} 式(\ref{eq1})において運動エネルギー項が
\begin{align}
  -\frac{\hbar^2}{2mr^2}\frac{\partial}{\partial r}\left(r^2 \frac{\partial}{\partial r}\right)\Psi_\text{radial}(r) \rightarrow
  -\frac{\hbar^2}{2mr}\frac{\partial^2}{\partial r^2} r\Psi_\text{radial}(r)
\end{align}
と変形できることを示せ。

\noindent
{\bf 問い2} 波動関数が
\begin{align}
  &\lim_{r\rightarrow 0}\Psi_\text{radial}(r)\rightarrow \ const \\
  &\lim_{r\rightarrow\infty}r^2\Psi_\text{radial}(r)\rightarrow 0
\end{align}   
となる場合、
\begin{align}
  |\Psi_\text{radial}(0)|^2=-\frac{2m}{\hbar^2}\left\langle\frac{d V}{dr}\right\rangle \label{eq3}
\end{align}   
が成立することを示せ。

\noindent
{\bf 問い3} 水素原子の1s波動関数に対して式(\ref{eq3})が成立することを確かめよ。



\end{document}
