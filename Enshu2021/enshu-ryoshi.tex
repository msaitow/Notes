\documentclass[12pt,pra,aps]{revtex4}
\usepackage[dvipdfmx]{graphicx}
\usepackage{float}
\usepackage{rotating}
\usepackage{array}
\usepackage{amsmath}
\usepackage{multirow}
\usepackage{setspace}
\usepackage{braket}
\usepackage{epstopdf}
\usepackage{moreverb}

\usepackage{color}                            
                                              
\newcommand{\red}[1]{\textcolor{red}{#1}}     
\newcommand{\blue}[1]{\textcolor{blue}{#1}}

\newcommand{\boxz}[1]{\boxed{\phantom{\text{#1}}}}
\newcommand{\boxa}[1]{\boxed{\phantom{}}}

%%\renewcommand{\baselinestretch}{2.0}

\renewcommand{\thefigure}{S\arabic{figure}}
\renewcommand{\thetable}{S\arabic{table}}

\begin{document}
\title{量子論と波動方程式}
\author{齋藤 雅明 \\ 量子化学研究室 \\ email: masa.saitow@chem.nagoya-u.ac.jp}

\maketitle

% Nakai & Ando, p. 87
\noindent
{\bf 問題1.} 次の共役二重結合をもつ1価の有機色素分子は、分子鎖の長さに対応する$k$の値 ($k=0\cdots 3$) が1増すごとに吸収する光の波長がおよそ100nmずつ長くなることが知られている。この現象は、$\pi$電子を自由電子模型により取り扱うことで理解される。このとき以下の問いに答えよ。

\noindent
{\bf 1.(a)} 長さ$L$の一次元の無限に高いポテンシャル中の電子のSchr\"odinger方程式を解き、エネルギー準位を求めよ。電子質量は$m_e$とせよ。自由電子模型の場合、$L$は共役長に対応し、$0.60+0.25k$ nmと与えられる。また$\pi$電子数は$4+2k$である。

\noindent
{\bf 1.(b)} $k=0$の色素分子に対してHOMOおよびLUMOの軌道エネルギーを求めよ。

\noindent
{\bf 1.(c)} 光吸収によるHOMO$\rightarrow$LUMO遷移を考える。$k=0$のときの吸収エネルギーおよび吸収波長を求めよ。

\noindent
{\bf 1.(d)} $k=1,2,3$についても1.(c)と同様の計算を行い、鎖長の増加に伴う吸収波長の変化を示せ。
    
% Nakai & Ando, p. 88    
\noindent
{\bf 問題2.} ポルフィリン化合物は様々な波長の可視光を吸収する。これは26個の$\pi$電子が分子平面全体に広がり、$\pi$共役するためである。この現象は$\pi$電子を二次元自由電子モデルで取り扱うことで理解できる。分子平面を正方形としたときの一辺の長さを$L$($=1.00$ nm)として、以下の問いに答えよ。

\noindent
{\bf 2.(a)} エネルギー準位が
\begin{align}
  E_{n_x n_y} = \frac{(n_x^2+n_y^2)h^2}{8m_eL^2}\ \ (n_x,n_y=1,2,\cdots)
\end{align}
と与えられることを示せ。

\noindent
{\bf 2.(b)} HOMO$\rightarrow$LUMO遷移の吸収波長を求めよ。

    
\end{document}
