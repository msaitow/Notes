\documentclass{jlreq}

\usepackage{luatexja-fontspec}
\usepackage{amsmath,amssymb,amsfonts}

\begin{document}

\noindent
\textbf{問1.}\\

\noindent
Formation of chemical bonds can be modeled by the following Hamiltonian of the two-state system
\begin{eqnarray}
    \mathbf{H}(\lambda)=
    \begin{pmatrix}
    E_1 & \lambda\Delta \\ 
    \lambda\Delta & E_2
    \end{pmatrix}
\end{eqnarray}
where $E_1$ and $E_2$ are the energy levels of the atomic orbitals (AOs) while $\lambda$ represents the strength of the interaction between two AOs, which serves as a perturbation.

\noindent
\textbf{(a).} Compute the exact eigenvalues of the Hamiltonian matrix $\mathbf{H}(\lambda)$ as a function of the strength of the perturbation $\lambda$.\\
\noindent
\textbf{(b).} Analyze the behaviour of the eigenvalues obtained in (a)  assuming $\lambda \ll |E_2-E_2|$.  Expand the energies up to second order in $\lambda$.\\ 
\noindent
\textbf{(c).} Energies obtained in \textbf{(b)} are regarded as those for the bonding and anti-bonding molecular orbitals (MOs). In which situation the energy gap between those two MOs are maximized?\\

\noindent
\textbf{Hint:} Calculate the Maclaurin expansion of $\sqrt{a+bx}$ up to the first order in $x$. Then, plug in $\lambda^2$ for $x$. 

\clearpage

\noindent
\textbf{問2.}\\

\noindent
The total $\pi$ electronic energy for  ring polyene with $N$ $\text{CH}_2$ units, $(\text{CH}_2)_N$, can be exactly solved with H\"uckel model and is given as follows:

\begin{eqnarray}
    E_\pi(N)=N\alpha+\frac{4\beta}{\sin(\pi/N)}
\end{eqnarray}

\noindent
\textbf{(a).} Calculate the resonance stabilization energy (RSE) for  $(\text{CH}_2)_N$.\\
\noindent
\textbf{(b).} Calculate the RSE as a function of $N$ in the limit of large $N$. Is the RSE a size-intensive or a extensive property?

\noindent
\textbf{Hint:} How does $\sin(x)$ behave for small $x$.

\clearpage

\noindent
\textbf{問3.}\\

\noindent
Calculate the ground state energy for a harmonic oscillator in the variational principle using the following trial function
\begin{eqnarray}
    \tilde{\phi}(x)=\exp(-\beta|x|) \ \ \ (\beta>0)
\end{eqnarray}  
where $\beta$ is the variational parameter. One can use the following formulas
\begin{eqnarray}
    &\frac{d|x|}{dx}=2\theta(x)-1 \\
    &\frac{d\theta(x)}{dx}=\delta(x) \\
    %&\int_{-\infty}^\infty \delta(x) f(x) dx = f(0) \\
    &\int_0^\infty \exp(-ax)x^n dx=\frac{n!}{a^{n+1}} \\
\end{eqnarray}  
where $\theta(x)$ and $\delta(x)$ are Heaviside step function and Dirac delta function, respectively.\\

%\noindent
%\textbf{Hint:} Use $(2\theta(x)-1)^2=1$.

\end{document}
	
